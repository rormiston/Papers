%\documentclass[letterpaper, aps, prd, twocolumn, superscriptaddress, showpacs, nofootinbib]{revtex4}
\documentclass[letterpaper, aps, prd, superscriptaddress, showpacs, nofootinbib]{revtex4}
\pdfoutput=1

%\usepackage{epsf}
\usepackage{graphicx}
\usepackage{bm}
%\usepackage{ltxgrid}
%\usepackage{multicol}
\usepackage{dcolumn}
\usepackage{color}
\usepackage[sort&compress]{natbib}
\usepackage[latin9]{inputenc}
\usepackage{url}
\usepackage{subfigure}
\usepackage{float}
\usepackage{longtable}
\usepackage{amssymb}
\usepackage{amsmath}
\usepackage{amsfonts}
% These lines activate hyperlinks in pdf; comment out if you don't
% like this.
% Put this package last
%\usepackage[bookmarks, bookmarksopen, bookmarksnumbered]{hyperref}
% Put this package after hyperref
%\usepackage[all]{hypcap}
% Don't use tt font for urls
\urlstyle{rm}

\newcommand\ligodoc{ADD}
\newcommand{\Msun}{M_{\odot}}

\newcommand{\todo}{\textcolor{red}}

\def\aj{Astron.\ J.}
\def\apj{Astrophys.\ J.}
\def\apjl{Astrophys.\ J. Lett.}
\def\apjs{Astrophys.\ J. Supp.\ Ser.}
\def\aa{Astron.\ Astrophys.}
\def\aap{Astron.\ Astrophys.}
\def\araa{Ann.\ Rev.\ Astron.\ Astroph.}
\def\physrep{Phys.\ Rep.}
\def\mnras{Mon.\ Not.\ Roy.\ Astron.\ Soc.}
\def\mmsun{M_\odot}
\def\prl{Phys.\ Rev.\ Lett.}
\def\prd{Phys.\ Rev.\ D}
\def\azh{Soviet Astron.}
\def\apss{Astrophys.\ Space Sci.}
\def\cqg{Class.\ Quantum Grav.}

%%%%%%%%%%%%%%%%%%%%%%%%%%%%%%%%%%%%%%%%%%%%%%%%%%%%%%%%%%%%%%%%%%%
%%%%%%%%%%%%%%%%%%%%%%%%%%%%%%%%%%%%%%%%%%%%%%%%%%%%%%%%%%%%%%%%%%%

\begin{document}

\title{Enhancing Gravitational-Wave Science with Machine Learning.}

\newcommand*{\ozgrav}{Universiry}
\affiliation{\ozgrav}

\author{Author Name}  \affiliation{\ozgrav}	

\date{\today}

%%%%%%%%%%%%%%%%%%%%%%%%%%%%%%%%%%%%%%%%%%%%%%%%%%%%%%%%%%%%%%%%%%%%
%%%%%%%%%%%%%%%%%%%%%%%%%%%%%%%%%%%%%%%%%%%%%%%%%%%%%%%%%%%%%%%%%%%%
\begin{abstract} 
\todo{If you contribute add your name here in any order so we can keep track of
the author list: Jade Powell, Coughlin$^2$, Elena Cuoco, Rich Ormiston.}

\end{abstract}

\maketitle

%%%%%%%%%%%%%%%%%%%%%%%%%%%%%%%%%%%%%%%%%%%%%%%%%%%%%%%%%%%%%%%%%%%%
%%%%%%%%%%%%%%%%%%%%%%%%%%%%%%%%%%%%%%%%%%%%%%%%%%%%%%%%%%%%%%%%%%%%
\section{Introduction}
\label{sec:intro}

The Advanced LIGO (aLIGO) \cite{aLIGO} and Advanced Virgo (AdVirgo) \cite{AdVirgo} gravitational-wave detectors have made the first direct detections of gravitational waves from compact binary systems \cite{2016PhRvX...6d1015A}. Although the LIGO and Virgo collaborations have shown they can successfully detect gravitational-wave signals and measure parameters of the source, there are still many challenges in the future of gravitational-wave astronomy. 

Gravitational-wave data is non-stationary, non-Gaussian, and contains many short duration noise artifacts that can contaminate signals or mask signals with a lower signal to noise ratio (SNR). Estimating the noise background of the detectors, and the astrophysical parameters of detections, is computationally expensive. As the detectors become more sensitive to lower frequencies, the duration of compact binary signals will be much longer, which will greatly increase the computational effort in the measurement of their parameters. Numerical relativity simulations of gravitational-wave signals also take months on the latest available computer facilities.  
 
In this study, we address how machine learning may be used to solve these issues, and other issues described later in the document, and enhance current tools designed for the analysis of gravitational-wave signals. Machine learning has gained popularity in gravitational wave science in recent years with advances made in areas including classification of noise transients \cite{powell:15, powell:16,PhysRevD.95.104059, 2017arXiv170607446G}, searches for compact binary systems \cite{PhysRevLett.120.141103, 2018PhLB..778...64G}, parameter estimation \cite{2012MNRAS.421..169G}, and citizen science projects \cite{gravityspy}. In this document, we give an overview of these studies as well more recent efforts in gravitational-wave machine learning science and outline how machine learning can improve gravitational wave science further in the future as current detectors reach their design sensitivity and future detectors become operational.   

In Section \ref{sec:mla}, we give an overview of machine learning and include a description of some of the current supervised and unsupervised methods. In Section \ref{sec:gw_inst}, we describe how machine learning can be used to improve gravitational-wave instrumentation. In Section \ref{sec:gw_data}, we describe how machine learning is used to improve the quality of gravitational-wave data. In Section \ref{sec:searches}, we describe how machine learning can be used to improve the sensitivity of searches for gravitational-wave signals. In Section \ref{sec:pe}, we show how we can speed up the estimation of signal parameters with machine learning. In Section \ref{sec:em}, we investigate if machine learning can be used to automate the process of sending alerts to electromagnetic partners. In Section \ref{sec:gw_astro}, we show how machine learning can help us understand the formation of sources of gravitational waves and the properties of the source populations. A discussion of current and future machine learning hardware and software is given in Section \ref{sec:hard_soft}, and a conclusion is given in Section \ref{sec:conclusion}.  

%%%%%%%%%%%%%%%%%%%%%%%%%%%%%%%%%%%%%%%%%%%%%%%%%%%%%%%%%%%%%%%%%%%%
%%%%%%%%%%%%%%%%%%%%%%%%%%%%%%%%%%%%%%%%%%%%%%%%%%%%%%%%%%%%%%%%%%%%
\section{Machine Learning}
\label{sec:mla}


\subsection{Unsupervised algorithms}

\subsection{Supervised algorithms}

%%%%%%%%%%%%%%%%%%%%%%%%%%%%%%%%%%%%%%%%%%%%%%%%%%%%%%%%%%%%%%%%%%%%
%%%%%%%%%%%%%%%%%%%%%%%%%%%%%%%%%%%%%%%%%%%%%%%%%%%%%%%%%%%%%%%%%%%%
\section{Gravitational Wave Instrumentation}
\label{sec:gw_inst}


The physics inherent to the design of the aLIGO detectors, including the shot noise of the laser light or thermal fluctuations of the mirror coatings and optic suspensions, sets their ultimate sensitivity~\citep{aLIGO}. On the other hand, the performance in the recent observing runs was mostly limited by non-fundamental noise sources, such as the instrumentation or control of the interferometer~\citep{AbEA2016g}.
Because the noise and sensitivity of the detectors determines our ability to both detect and extract astrophysical information from signals, it is important to reduce the effect of these noise sources.
Towards that end, the aLIGO detectors record a large numbers of signals related to numerous subsystems that control different aspects of the instrument and monitor its state. These can include instrument monitors such as photodetectors and environmental sensors such as seismometers.
The idea is that these monitoring channels measure the sources of noise that will couple into the interferometer, and thereby can by used to diagnose and mitigate such couplings.

A common technique for using witness sensors (such as the environmental channels) to remove noise from a target channel (such as the gravitational-wave strain) is Wiener filtering~\citep{Vas2001,Say2003}, a multiple-input single-output (MISO) algorithm which naturally takes into account inter-channel correlations~\citep{DeEA2012,CoMu2016}.
Because Wiener filtering is limited to subtracting linear couplings, there is significant interest in developing regression techniques sensitive to nonlinear couplings. 
As many of these nonlinear couplings are either unknown or poorly modeled, it is important to develop methods that do not require precise knowledge of the coupling method.
The hope is that MLAs can provide the subtraction algorithms we currently lack for nonlinear couplings.

Going forward, development of MLAs for the possibility of off-line subtraction of past data would have significant benefits. 
For example, removal of further noise would make it possible for marginal event candidates to be promoted to fully confident detections, or to further constrain events already studied.
MLA regression techniques running in an online environment would facilitate the use of cleaned data in the online gravitational-wave search pipelines, increasing the number of detections in low-latency, making multi-messenger followup easier.
Finally, given the significant cost of the gravitational-wave detectors, increases in range from software can provide significant cost savings relative to the same requirements in hardware.

%%%%%%%%%%%%%%%%%%%%%%%%%%%%%%%%%%%%%%%%%%%%%%%%%%%%%%%%%%%%%%%%%%%%
%%%%%%%%%%%%%%%%%%%%%%%%%%%%%%%%%%%%%%%%%%%%%%%%%%%%%%%%%%%%%%%%%%%%
\section{Gravitational Wave Data Quality}
\label{sec:gw_data}

With the thousands of environmental monitoring sensors at LIGO, it is impossible for scientists to monitor them all. To this end, LIGO has recently launched a website for citizen scientists to classify detector transients and potential gravitational-wave signals \cite{gravityspy}. This ambitious project seeks to leverage the advantages of citizen science and machine learning to design a socio-computational system with which to analyze and characterize transients in gravitational-wave data, which will improve the effectiveness of gravitational-wave searches. The citizen scientists classify data transients into categories, which the machine learning algorithms take advantage of for ``learning.'' The idea is that as the citizen scientists make identifications, these classifications are provided to the gravitational wave data analysis pipelines in order to make use of them. In addition, as the citizen scientists identify new categories of transients, the machine learning algorithms are automatically updated to use these in their classifications. Due to the size of the data sets employed, with databases exceeding 200,000 classified transients, the machine learning training benefits significantly from parallel, GPU-based training.

%%%%%%%%%%%%%%%%%%%%%%%%%%%%%%%%%%%%%%%%%%%%%%%%%%%%%%%%%%%%%%%%%%%%
%%%%%%%%%%%%%%%%%%%%%%%%%%%%%%%%%%%%%%%%%%%%%%%%%%%%%%%%%%%%%%%%%%%%
\section{Signal Searches}
\label{sec:searches}

In this section, we describe how machine learning can be used to enhance the searches for 
different types of gravitational wave signals. 

\subsubsection{Matched filter searches}

\subsubsection{Burst searches}

\subsubsection{Continuous wave searches}

\subsubsection{Stochastic background}

%%%%%%%%%%%%%%%%%%%%%%%%%%%%%%%%%%%%%%%%%%%%%%%%%%%%%%%%%%%%%%%%%%%%
%%%%%%%%%%%%%%%%%%%%%%%%%%%%%%%%%%%%%%%%%%%%%%%%%%%%%%%%%%%%%%%%%%%%
\section{Parameter Estimation}
\label{sec:pe}

To understand the astrophysics behind sources of gravitational waves, it is essential to accurately 
measure the parameters of the source. This is currently achieved using the \texttt{LALInference} \cite{veitch:15}
tool designed for Bayesian parameter estimation and model selection. A Bayesian framework allows us to calculate posterior probability density functions (PDFs) for the parameters of gravitational-wave signals. It also allows us to calculate the evidence for different models which can be used for model selection. Bayesian evidence is computationally costly. In the case of compact binary signals this is due to the high number of signal parameters $(\sim 15)$, the process of generating waveforms, and the SNR and length of the signal being analyzed. In \texttt{LALInference}, the computational issues are addressed using either nested sampling \cite{2010PhRvD..81f2003V} or Markov Chain Monte Carlo (MCMC). 

To estimate a posterior distribution, MCMC techniques work by stochastically wandering though a parameter space, distributing samples that are proportional to the density of the posterior.  The \texttt{LALInference} implementation of MCMC uses the Metropolis-Hastings algorithm that requires a proposal density function to generate a new sample that can only depend on the current sample. The efficiency depends on the choice of the proposal density function. 

Nested sampling is used to calculate the Bayesian evidence and can also produce PDFs for the signal parameters. Nested sampling transforms the mulit-dimensional evidence integral into a one dimensional integral over the prior volume. First a set of live points are distributed over the entire prior. The point with the lowest likelihood is then removed and replaced with a point with a higher likelihood and continues until some stopping condition is reached. Posterior samples can then be produced by re-sampling the chain of removed points and current live points according to their posterior probabilities.

Bambi \cite{2012MNRAS.421..169G} and multinest \cite{2009MNRAS.398.1601F}.

%%%%%%%%%%%%%%%%%%%%%%%%%%%%%%%%%%%%%%%%%%%%%%%%%%%%%%%%%%%%%%%%%%%%
%%%%%%%%%%%%%%%%%%%%%%%%%%%%%%%%%%%%%%%%%%%%%%%%%%%%%%%%%%%%%%%%%%%%
\section{Electromagnetic follow-up}
\label{sec:em}


%%%%%%%%%%%%%%%%%%%%%%%%%%%%%%%%%%%%%%%%%%%%%%%%%%%%%%%%%%%%%%%%%%%%
%%%%%%%%%%%%%%%%%%%%%%%%%%%%%%%%%%%%%%%%%%%%%%%%%%%%%%%%%%%%%%%%%%%%
\section{Gravitational Wave Astrophysics}
\label{sec:gw_astro}

Multiple detections of gravitational-wave signals will allow us to begin population studies of their sources.
In the case of compact binaries, measuring properties such as their mass and spin distributions could allow 
us to determine their formation mechanism \cite{0264-9381-27-11-114007, 2017Natur.548..426F, 2017MNRAS.471.2801S, 2015ApJ...810...58S, 0264-9381-34-3-03LT01, 2017PhRvD..95l4046G, 2018arXiv180102699T}.

%%%%%%%%%%%%%%%%%%%%%%%%%%%%%%%%%%%%%%%%%%%%%%%%%%%%%%%%%%%%%%%%%%%%
%%%%%%%%%%%%%%%%%%%%%%%%%%%%%%%%%%%%%%%%%%%%%%%%%%%%%%%%%%%%%%%%%%%%
\section{Hardware and Software}
\label{sec:hard_soft}

%%%%%%%%%%%%%%%%%%%%%%%%%%%%%%%%%%%%%%%%%%%%%%%%%%%%%%%%%%%%%%%%%%%%
%%%%%%%%%%%%%%%%%%%%%%%%%%%%%%%%%%%%%%%%%%%%%%%%%%%%%%%%%%%%%%%%%%%%
\section{Conclusions}
\label{sec:conclusion}

%%%%%%%%%%%%%%%%%%%%%%%%%%%%%%%%%%%%%%%%%%%%%%%%%%%%%%%%%%%%%%%%%%%%
%%%%%%%%%%%%%%%%%%%%%%%%%%%%%%%%%%%%%%%%%%%%%%%%%%%%%%%%%%%%%%%%%%%%

\bibliographystyle{apsrev}

\bibliography{bibfile}

\end{document}















